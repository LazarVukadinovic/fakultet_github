\documentclass[a4, 11pt, leqno]{article}
\usepackage{geometry}
\usepackage{multirow}
\usepackage{enumerate}

\geometry{
    left=20mm,
    right=30mm,
    top=30mm,
    bottom=20mm
}

\def\dj{d\kern-0.4em\char"16\kern-0.1em}
\def\Dj{mbox{\raise0.3ex\hbox{-}\kern-0.4em D}}

\author{Pera Peri\'{c}}
\title{Doma\' ci}

\begin{document}

  \maketitle
  \thispagestyle{empty}
  \newpage

  \centerline{\bf Doma\' ci zadatak}
  \medskip
  \centerline{28.04.2018.}

  \vskip1cm

  \begin{enumerate}

    \item \begin{enumerate}[a)]
            \item Pokazati da se sistem linearnih jedna\v cina $Ax=b$, gde su
                $$A=\left[
                    \begin{array}{ccc}
                      10 & 1 & 0 \\
                      1 & 4 & -1 \\
                      1 & 2 & -5 \\
                    \end{array}
                \right] ,\quad
                  b=\left[
                    \begin{array}{ccc}
                      11 & 6 & 11 \\
                    \end{array} \right]^T,$$
              mo\v ze re\v siti i {\bf Jacobijevom} i {\bf metodom Nekrasova}.

            \item Ako je $x^{*}$ ta\v cno re\v senje jedna\v cine $x=Bx+\beta$, $B=B_1+B_2$, i ako je $\|B\|\leq q<1$ tada va\v zi nejednakost
                $$\|x^{(k)}-x^*\|< \frac{\|B_2\|}{1-\|B\|} \|x^{(k)}-x^{(k-1)}\|,k\in \mathbf{N},$$ % \mathbb{N}
                gde se niz $(x^{(k)})$ generi\v se pomo\' cu
                \begin{equation}
                \label{itproces}
                  x^{(k)}=B_1x^{(k)}+B_2x^{(k-1)}+\beta.
                \end{equation}
             Dokazati.
           \end{enumerate}

    (Formula (\ref{itproces}) se naziva {\bf iterativni proces}.)


    \item  Odrediti parametre i ostatak u kvadraturnoj formuli Gaussovog tipa
      $$\int\limits_{-1}^1\, p(x)f(x)\, dx=A_1f(x_1)+A_2f(x_2)+A_3f(x_3)+R_3(f)\, ,$$
      ako je te\v zinska funkcija $p(x)=x(1-x^2)$.\\

    \item Ispitati neprekidnost slo\v zene funkcije $y=f(t)$ gde je $t=g(x)$ ako je \[f(t)=\left\{
       \begin{array}{ll}
       t, & \; 0<t< 1\;\; \\
       2-t, & \; 1<t<2.\;
       \end{array}\right.\quad\mbox{ i} \quad g(x)=\left\{
       \begin{array}{ll}
            x,   &\; x\in Q ,\;\; \\
            2-x, & \; x\in I, \quad 0<x<1.\;
       \end{array}\right.\]

  \end{enumerate}
\begin{center}
$ $\\ $ $\\
    \begin{tabular}{|c|c|ccccc|}
        \hline
        \multicolumn{2}{|c|}{\multirow{2}{*}{}}             & \multicolumn{4}{c|}{Rank}                  & \multirow{2}{*}{Total} \\ \cline{3-6}
        \multicolumn{2}{|c|}{}                              & A  & B  & C  & \multicolumn{1}{c|}{Other} &                        \\ \hline
        \multirow{2}{*}{Type} & type 1                      & 10 & 21 & 6  & \multicolumn{1}{l}{3}                          & 40                     \\ \cline{2-7}
                              & type 2                      & 8  & 14 & 5  & 2                          & 29                     \\ \hline
        \multicolumn{2}{|c|}{\textbf{Total}}                & 18 & 35 & 11 & \multicolumn{1}{r}{5}                          & 69                     \\ \hline
    \end{tabular}
\end{center}
\end{document} 