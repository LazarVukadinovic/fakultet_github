\documentclass[11pt, a4paper, leqno]{article}
\usepackage{geometry}
\usepackage{enumerate}
\usepackage{multirow}
\usepackage{multicol}

\geometry{
    left=20mm,
    right=30mm,
    top=30mm,
    bottom=20mm
}

\title{Doma\' ci}
\author{Pera Peric}
\date{January 10, 2021}

\begin{document}
    \maketitle
    \thispagestyle{empty}
    \newpage
    \begin{center}
        \textbf{Domaci zadatak}\\
        \medskip
        \date{28.04.2018.}
    \end{center}
    \vskip1cm
    
    \begin{enumerate}
        \item \begin{enumerate}[a)]
            \item Pokazati da se sistem linearnih jednacina $Ax=b$, gde su
            $$
                A = \left[ 
                    \begin{array}{ccc}
                        10 & 1 & 0\\
                        1 & 4 & -1\\
                        1 & 2 & -5\\
                    \end{array}
                \right], \quad b=\left[ 
                    \begin{array}{ccc}
                        11 & 6 & 11\\
                    \end{array}
                 \right]^T,
            $$
            moze resiti i \textbf{Jacobijevom i metodom Nekrasova.}
        \item Ako je $x^*$ tacno resenje jednacine $x=Bx+ \beta, B = B_1 + B_2,$ i ako je $\|B\| \leq q < 1$ tada vazi nejednakost
            $$
                \|x^{(k)}-x^*\| < \frac{\|B_2\|}{1-\|B\|}\|x^{(k)}-x^{(k-1)}\|, k \in \mathbf{N}
            $$
            gde se niz $(x^{(k)})$ generise pomocu
            \begin{equation}
            \label{itproces}
                x^{(k)}=B_1x^{(k)}+B_2x^{(k-1)}+ \beta .
            \end{equation}
            Dokazati.
        \end{enumerate}
        (Formula (\ref{itproces}) se naziva \textbf{iterativni proces.})
        \item Odrediti parametre i ostatak u kvadratnoj formuli Gaussovog tipa
        $$
            \int\limits_{-1}^1{p(x)f(x)dx}=A_1f(x_1)+A_2f(x_2)+A_3f(x_3)+R_3(f),
        $$
        ako je tezinska funkcija $p(x)=x(1-x^2)$.\\
        \item Ispitati neprekidnost slozene funkcije $y=f(t)$ gde je $t=g(x)$ ako je
        $$
            f(t)=\left\{ 
                \begin{array}{ll}
                    t, & \quad 0<t<1\;\;\\
                    2-t,& \quad 1<t<2.\;
                \end{array}
             \right.
             \quad \mbox{i} \quad g(x)= \left\{
                \begin{array}{ll}
                    x, & \; x \in Q, \;\; \\
                    2-x,& \; x \in I \quad 0<x<1.\;
                \end{array}
             \right.
        $$
    \end{enumerate}
    \medskip
    \begin{center}
        \begin{tabular}{|c|c|cccc|c|}
        \hline
            \multicolumn{2}{|c|}{\multirow{2}{*}{}} & \multicolumn{4}{|c|}{Rank} & \multirow{2}{*}{Total}\\
            \cline{3-6}
            \multicolumn{2}{|c|}{} & A & B & C & Other & \\
            \hline
            \multirow{2}{*}{Type} & type 1 & 10 & 21 & 6 & \multicolumn{1}{l}{3} & 40\\
            \cline{2-7}
            & type 2 & 8 & 14 & 5 & \multicolumn{1}{c}{2} & 29\\
            \hline
            \multicolumn{2}{|c|}{Total} & 18 & 35 & 11 & \multicolumn{1}{r}{5} & 69\\
            \hline
        \end{tabular}
    \end{center}
\end{document}