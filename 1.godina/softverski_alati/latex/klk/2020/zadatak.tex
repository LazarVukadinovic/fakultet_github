\documentclass[]{article}
\usepackage{multirow}
\usepackage{multicol}
\usepackage{enumerate}
\usepackage{indentfirst}

\title{Prvi kolokvijum}
\date{April 11, 2021}

\renewcommand{\contentsname}{Sadr\v zaj}

\begin{document}
    \maketitle
    \thispagestyle{empty}
    \newpage
    \centerline{Lazar Vukadinovi\' c}
    \section{Uvod}
    \begin{center}
        \begin{tabular}{|l|l|l|l|}
        \hline
        \textbf{Level 1} & \textbf{Level 2} & \textbf{Level 3} & \textbf{Info}\\
        \hline
        \multirow{6}{*}{System} & \multirow{4}{*}{System Apps} & \multirow{3}{*}{SystemEnv} & App Test\\
        \cline{4-4}
        & & & App Memory\\
        \cline{4-4}
        & & & App Test\\
        \cline{3-4}
        & & \multicolumn{2}{|r|}{\itshape{SystemEnv2}}\\
        \cline{2-4}
        & \multirow{2}{*}{System Memory} & \multirow{2}{*}{MemoryTest} & Memory Fune\\
        \cline{4-4}
        & & & Apes Test\\
        \hline
        \end{tabular}
    \end{center}
    \begin{enumerate}
        \item \textbf{Date su funkcije} $f(x)$ i $g(x)$ u zapisu:
        $$
            f(x)= \left\{
                \begin{array}{lr}
                    0, & \, x<0\\
                    \frac{f_1}{h}x, & \, 0 \leq x \leq h\\
                    0, & \, x>h
                \end{array}
             \right.
             , \quad g(x)=\left\{
                \begin{array}{lr}
                    g_2, & \, x<0\\
                    \frac{g_1}{h}x, & \, 0 \leq x \leq h\\
                    \frac{g_1}{h}x, & \, 0
                \end{array}
             \right.
        $$
        Za vi\v se detalja pogledati [\ref{1}]
        \item Loss funkcija za Mask RCNN:
        \begin{equation}
        \label{1}
            \lambda_{mask}=-\frac{1}{m^2} \sum\limits_{1 \leq i,j \leq m}{[y_{ij}^1 \log{y_{ij}}+(1-y_{ij}^1) \log{(1-y_{ij}^1)}]}
        \end{equation}
        Za vi\v se informacija pogledati [\ref{2}]
        \item Data je jedna\v cina:
        \begin{equation}
        \label{2}
            E=\int\limits_0^{\infty}{\frac{x^3}{e^x-1}dx}=\int\limits_0^{\infty}{(x^3 \sum\limits_{n=1}^{\infty}{e^{-nx}})dx}=\sum\limits_{n=1}^{\infty}{x^3 e^{-nx}dx}
        \end{equation}
        \begin{enumerate}[a)]
            \item Prona\' ci E iz jedna\v cine (\ref{2}).
            \item Prokomentarisati dobijeni rezultat.
        \end{enumerate}
    \end{enumerate}
    \section{Modelovanje}
    \begin{multicols}{2}
        {\itshape Latent Dirichlet Allocation}\footnote{pogledati literaturu za vi\v se detalja}, nadalje LDA, je najjednostavniji pristup problemu modelovanja tema, i njegova primena je predmet ovog rada. Osnovna karakteristika LDA algoritma je mogu\' cnost \textbf{izdvajanja} tema koje su prisutne u nekoj kolekciji dokumenata bez ikakvog dodatnog znanja. Dakle, primenom LDA-a mogu\' ce je otkriti teme o kojima govori zadati skup dokumenata a da se pritom nikakvo dodatno ekspertsko znanje ne uklju\v cuje. Polazna pretpostavka LDA-a je da svaki dokument u kolekciji dokumenata "govori o" vi�e tema. Opravdanost ove pretpostavke bi\' ce ilustrovana na nekoliko primera. Dobro je poznat roman Branka \' Copica "Orlovi rano lete". Ukoliko bi neko ko nije pro\v citao ovu knjigu �eleo da zna "o \v cemu se radi" u njoj, najverovatnije bi dobio odgovor da je u pitanju knjiga koja se bavi do\v zivljajima grupe de\v caka na po\v cetku Drugog svetskog rata. Iako je to naj\v siri okvir romana, u njemu su prisutne i teme o ljubavi, dru\v zenju, prijateljstvu, ratu, pustolovinama itd. Prema tome, roman, op\v ste gledano, obuhvata vi\v se tema, ali se sa nekoliko njih intenzivno bavi.
    \end{multicols}
    \section{Zaklju\v cak}
    Naravno, postoje re\v ci koje se mogu svrstati u vi\v se od jedne teme. Takve re\v ci bi bile obojene sa dve ili vi�e boja, ali zbog preglednosti slike, takvi slu\v cajevi su izostavljeni.
    \newpage
    \thispagestyle{empty}
    \begin{thebibliography}{99}
        \bibitem{mell}  Mell,  Peter,  and  Tim  Grance.  The  NIST  definition  of  cloud computing. (2011)
        \bibitem{buyya}  Buyya, Rajkumar, et al. "Cloud computing and emerging IT platforms: Vision, hype, and reality for delivering computing as the 5th utility." Future Generation computer systems 25.6 (2009): 599-616.
    \end{thebibliography}
    \newpage
    \thispagestyle{empty}
    \tableofcontents
\end{document}